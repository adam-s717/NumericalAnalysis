


\documentclass{article}

\usepackage{fancyhdr}
\usepackage{extramarks}
\usepackage{amsmath}
\usepackage{amsthm}
\usepackage{amsfonts}
\usepackage{tikz}
\usepackage[plain]{algorithm}
\usepackage{algpseudocode}
\usepackage{scrextend}
\usepackage{systeme}
\usepackage{mathtools}
\usepackage{amssymb}
\usetikzlibrary{automata,positioning}
\usepackage{graphicx}
\usepackage{bm}




%
% Basic Document Settings
%

\topmargin=-0.45in
\evensidemargin=0in
\oddsidemargin=0in
\textwidth=6.5in
\textheight=9.0in
\headsep=0.25in

\linespread{1.1}

\pagestyle{fancy}
\lhead{Adam Sanchez}
\chead{APPM 4515-001: Homework 8}
\cfoot{ }
\rfoot{\thepage}



%
% Homework Problem Environment
%
% This environment takes an optional argument. When given, it will adjust the
% problem counter. This is useful for when the problems given for your
% assignment aren't sequential
%


%
% Homework Details
%   - Title
%   - Due date
%   - Class
%   - Section/Time
%   - Instructor
%   - Author
%

\newcommand{\hmwkTitle}{Homework\ \#8}
\newcommand{\hmwkDueDate}{November 25, 2020}
\newcommand{\hmwkClass}{APPM 4515-001}
\newcommand{\hmwkClassTime}{CU Boulder}
\newcommand{\hmwkClassInstructor}{Dr. Meyer}
\newcommand{\hmwkAuthorName}{\textbf{Adam Sanchez}}

%
% Title Page
%

\title{
    \vspace{2in}
    \textmd{\textbf{\hmwkClass:\ \hmwkTitle}}\\
    \normalsize\vspace{0.1in}\small{Due\ on\ \hmwkDueDate\ at 11:59pm}\\
    \vspace{0.1in}\large{\textit{\hmwkClassInstructor\ \hmwkClassTime}}
    \vspace{3in}
}

\author{\hmwkAuthorName}
\date{}

\renewcommand{\part}[1]{\textbf{\large Part \Alph{partCounter}}\stepcounter{partCounter}\\}


\begin{document}

\maketitle

\pagebreak

\textbf{Problem 1}

\begin{eqnarray*}
y' &=& \lambda y\\
k_1 &=& h\lambda w_i\\
k_2 &=& h\lambda \left(w_i + \frac{h\lambda w_i}{2}\right) = h\lambda w_i + \frac{h^2\lambda^2 w_i}{2}\\
k_3 &=& h\lambda \left(wi +\frac{h\lambda w_i + \frac{h^2\lambda^2 w_i}{2}}{2}\right) = h\lambda w_i + \frac{h^2\lambda^2 w_i}{2}+\frac{h^3\lambda^3 w_i}{4}\\
k_4 &=& h\lambda\left(h\lambda w_i + \frac{h^2\lambda^2 w_i}{2}+\frac{h^3\lambda^3 w_i}{4}\right)
\end{eqnarray*}

\begin{eqnarray*}
w_{i+1} &=& w_i +\frac{1}{6}\left[ h\lambda w_i + 2h\lambda w_i + \frac{2h^2\lambda^2 w_i}{2} + 2h\lambda w_i + \frac{2h^2\lambda^2 w_i}{2}+\frac{2h^3\lambda^3 w_i}{4} + h\lambda w_i + h^2\lambda^2 w_i+ \frac{h^3\lambda^3 w_i}{2} + \frac{h^4\lambda^4 w_i}{4}\right]\\
&=& w_i + \frac{w_i}{6}\left[h\lambda + 2h\lambda + h^2\lambda^2 + 2h\lambda + h^2\lambda^2 + \frac{h^3\lambda^3}{2}  + h\lambda + h^2\lambda^2 + \frac{h^3\lambda^3}{2} + \frac{h^4\lambda^4}{4}\right]\\
&=& \left[1 + h\lambda + \frac{h^2\lambda^2}{2}+ \frac{h^3\lambda^3}{6}+\frac{h^4\lambda^4}{24}\right]w_i
\end{eqnarray*}

\begin{eqnarray*}
&y'' +4y' +13y = 0\\
&\text{ansatz } y = e^{rx}\\
&r^2+4r+13 = 0\\
&r = -2 \pm 3i\\
&\implies y(x) = C_1e^{2x}\cos{3x}+C_2e^{-2x}\sin{3x}
\end{eqnarray*}

\begin{eqnarray*}
|Q(h\lambda)| &=& \left|1-2(\frac{3}{4})+ \frac{1}{2}\left(-2\frac{3}{4}\right)^2 + \frac{1}{6}\left(-2\frac{3}{4}\right)^3 + \frac{1}{24}\left(-2\frac{3}{4}\right)^4\right|\\
&=&\left|\frac{35}{128}\right| < 1
\end{eqnarray*}


\end{document}