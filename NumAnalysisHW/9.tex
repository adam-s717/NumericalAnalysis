\documentclass{article}

\usepackage{fancyhdr}
\usepackage{extramarks}
\usepackage{amsmath}
\usepackage{amsthm}
\usepackage{amsfonts}
\usepackage{tikz}
\usepackage[plain]{algorithm}
\usepackage{algpseudocode}
\usepackage{scrextend}
\usepackage{systeme}
\usepackage{mathtools}
\usepackage{amssymb}
\usetikzlibrary{automata,positioning}
\usepackage{graphicx}
\usepackage{bm}




%
% Basic Document Settings
%

\topmargin=-0.45in
\evensidemargin=0in
\oddsidemargin=0in
\textwidth=6.5in
\textheight=9.0in
\headsep=0.25in

\linespread{1.1}

\pagestyle{fancy}
\lhead{Adam Sanchez}
\chead{APPM 4515-001: Homework 5}
\cfoot{ }
\rfoot{\thepage}



%
% Homework Problem Environment
%
% This environment takes an optional argument. When given, it will adjust the
% problem counter. This is useful for when the problems given for your
% assignment aren't sequential
%


%
% Homework Details
%   - Title
%   - Due date
%   - Class
%   - Section/Time
%   - Instructor
%   - Author
%

\newcommand{\hmwkTitle}{Homework\ \#6}
\newcommand{\hmwkDueDate}{October 19, 2020}
\newcommand{\hmwkClass}{APPM 4515-001}
\newcommand{\hmwkClassTime}{CU Boulder}
\newcommand{\hmwkClassInstructor}{Dr. Meyer}
\newcommand{\hmwkAuthorName}{\textbf{Adam Sanchez}}

%
% Title Page
%

\title{
    \vspace{2in}
    \textmd{\textbf{\hmwkClass:\ \hmwkTitle}}\\
    \normalsize\vspace{0.1in}\small{Due\ on\ \hmwkDueDate\ at 11:59pm}\\
    \vspace{0.1in}\large{\textit{\hmwkClassInstructor\ \hmwkClassTime}}
    \vspace{3in}
}

\author{\hmwkAuthorName}
\date{}

\renewcommand{\part}[1]{\textbf{\large Part \Alph{partCounter}}\stepcounter{partCounter}\\}


\begin{document}

\maketitle

\pagebreak

\textbf{Problem 1}

From figure 1 we can tell that: 

\begin{eqnarray*}
y'' + 5y' +6y = \cos{t} \text{; } y(0) = 1 \text{; } y'(0) = 0\\
\text{Let } y_h \text{ be the solution to } y'' + 5y' +6y = 0\\
\text{ansatz } y_h = e^{rt} \text{ where r: }\\
r^2 + 5r + 6 = 0\\
(r+2)(r+3) = 0 \\
r = -2 \text{; } r = -3 \\
y_h = C_1e^{-2t}+C_2e^{-3t}\\
\text{Let } y_p \text{ be the particular solution to the ODE: }\\
\text{ansatz } y_p = B_0\sin{t}+A_0\cos{t}\\
\implies y_p' = B_0\cos{t} -A_0\sin{t}\\
\implies y_p'' = -B_0\sin{t}-A_0\cos{t}\\
\text{Substituting these into the ODE: }\\
-B_0\sin{t}-A_0\cos{t} + 5B_0\cos{t} -5A_0\sin{t} + 6B_0\sin{t}+6A_0\cos{t} = \cos{t}\\
\left(-A_0 + 5B_0+6A_0 \right)\cos{t} + \left(-B_0-5A_0+6B_0\right) \sin{t} = \cos{t} + 0 \sin{t}\\
\implies -A_0 + 5B_0+6A_0 = 1 \text{; } -B_0-5A_0+6B_0 = 0\\
\implies A_0 = \frac{1}{10} \text{; } B_0 = \frac{1}{10}\\
\text{Thus } y_p = \frac{1}{10}\sin{t} + \frac{1}{10}\cos{t}\\
\text{So the general solution is: }\\
y(t) = C_1e^{-2t}+C_2e^{-3t} + \frac{1}{10}\sin{t} + \frac{1}{10}\cos{t}\\
\text{Now implementing our initial conditions: }\\
y'(t) = -2C_1e^{-2t}-3C_2e^{-3t} + \frac{1}{10}\cos{t} - \frac{1}{10}\sin{t}\\
y(0) = C_1e^{-2*0}+C_2e^{-3*0} + \frac{1}{10}\sin{0} + \frac{1}{10}\cos{0} = 1\\
 \implies  C_1 +C_2 +\frac{1}{10} = 1 \textbf{ (1) }\\
y'(0) = -2C_1e^{-2*0}-3C_2e^{-3*0} + \frac{1}{10}\cos{0} - \frac{1}{10}\sin{0} = 0\\
\implies 2C_1 + 3C_2 = \frac{1}{10}\textbf{ (2) }\\
\text{From } \textbf{(1)} \text{ and } \textbf{(2)} \text{ we can conclude: }\\
C_1 = \frac{13}{5} \text{; } C_2 -\frac{17}{10}\\
\text{Therefore our solution is: }\\ 
y(t) = \frac{13}{5}e^{-2t}-\frac{17}{10}e^{-3t}+\frac{1}{10}\sin{t}+\frac{1}{10}\cos{t}
\end{eqnarray*}

So the radius of $S^{n-2}(x_1,r)$ is $\sqrt{1-\epsilon^2}$.
\\~\\

\textbf{Problem 2}

$Vol(B^n(x_1,r)) = \frac{n^{n/2}}{\Gamma(n/2 +1})\sqrt{1-\epsilon^2}^n$

$Vol(B^n(0,1)) = \frac{n^{n/2}}{\Gamma(n/2 +1})$

So it follows that $Vol(B^n(x_1,r)) = Vol(B^n(0,1))\sqrt{1-\epsilon^2}^n = Vol(B^n(0,1))(1-\epsilon^2)^\frac{n}{2}$
\\~\\

\textbf{Problem 3}

Graphically we can see that for every $\gamma \in C(k)$, $\gamma \in B^n(x_1,r)$ So this tells us that $C(k) \subset B^n(x_1,r)$. 

Formally we choose a $\gamma \in C(k)$ and show that because the radius of $B^n(x_1,r)$ is greater than or equal 

to the radius of $C(k)$, $\gamma \in B^n(x_1,r)$. 
\\~\\

\textbf{Problem 4}

We know from problem 3 that $Vol(C(k)) \leq Vol(B^n(x_1,r))$ because it is a subset, 

(note $1+x \leq e^x \implies (1+x)^a \leq (e^x)^a$), so it follows that:  

\begin{eqnarray*}
y'' + 5y' +6y = \cos{t}\\
y'' = \cos{t} - 5y' - 6y\\
\text{Let } y_1 = y \text {; } y_2 = y'\\
\text{So our system is: }\\
y_1' = y_2 = u'\\
y_2' = y''  = -6y_1-5y_2+cos(t) = -6u-5v+\cos{t}=v'
\end{eqnarray*}



\textbf{Problem 5}


\begin{eqnarray*}
\mu(k(x_0,\epsilon)) &\leq& Vol(C(k))\\
&\leq& Vol(B^n(0,1))e^{-\frac{n\epsilon^2}{2}}\\
&\leq& e^{-\frac{n\epsilon^2}{2}}\\
\end{eqnarray*}


\textbf{Problem 6}

This is really close to what we got for Levy's bound, but not quite a good.I believe Levy's bound would 

have been $\sqrt{\frac{\pi}{8}}e^\frac{-(n-2)\epsilon^2}{2}$.

\end{document}